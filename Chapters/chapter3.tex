%!TEX root = ../template.tex
%%%%%%%%%%%%%%%%%%%%%%%%%%%%%%%%%%%%%%%%%%%%%%%%%%%%%%%%%%%%%%%%%%%%
%% chapter3.tex
%% NOVA thesis document file
%%
%% Chapter with a short laext tutorial and examples
%%%%%%%%%%%%%%%%%%%%%%%%%%%%%%%%%%%%%%%%%%%%%%%%%%%%%%%%%%%%%%%%%%%%
\chapter{Approach of Elaboration Phase}
\label{cha:approach_of_elaboration_phase}


\section{Addressing the objectives and contributions} % (fold)
\label{sec:document_structure}


fulcral points: 1) descentralization of the processing and storage of data (increase availability)
2) reduce TCB size as much as possible on each node
3) privacy of data (on storing and processing)


\section{System Model Overview} % (fold)
\label{sec:dealing_with_bibliogrpahy}

model composed in four main components:

1- Client (app)
2- Processing layer (redis)
3- Memory layer (ram)
4- Storage layer (RAID of disks) - increased redundancy


\section{Adversary Model} % (fold)
\label{sec:inserting_tables}

We prevent interventions made by System Administrators on Nodes from having any
effect on the computations of users. This means we want to avoid having System Admin-
istrators on a node with computations running, however we want to avoid interrupting

the computations or degrading its throughput. If interruptions or degradation happen,
we have to make sure these are as small as possible.
We assume all nodes in the infrastructure have TPMs in their hardware architectures,
and these work correctly, properly providing the functions as we described in the Chapter
2. Only the TPM possesses the proposer TPM’s private keys in validated certification
chains. Any recipient of the booted configuration (composed by the BIOS configuration,
the bootloader, Host OS and all the relevant software components in the Host OS level)
can use the TPM’s public key and the certification chain (ending in a root of trust) to
verify the related signatures.
We assume that all nodes in the infrastructure are equipped with Intel SGX enabled

CPUs and the necessary firmware support at the Host OS support level, which is con-
tained in the TCB of the solution.

PaaS software or SaaS applications should be deployed with minimal interaction (e.g.
using Preboot eXecution Environment) as stock images and, is assumed correct as far
as promised functionality and security are concerned. As we will target on Dockerized
images, we consider that Docker does not corrupt applications and is able to contain
them, with fully execution isolation. Current cryptographic protocols and standards, as


\section{Isolation and Containerization} % (fold)
\label{sec:importing_images}

% section importing_images (end)

\section{System Generalization} % (fold)



\section{Implementation Guidelines} % (fold)

KeyValue Store technology (Redis)

Intel SGX

Cloud Infrastructure (OVH)

OS library virtualization (Graphene SGX)

Containerized OS virtualization (Graphene Secured Container - GSC)

Miscellaneous: - Java
			   - Jedis (Redis client)
			   - Benchmarks: 1) redisbenchmark
			   				 2) Yahoo! Cloud Serving Benchmark (YCSB)

\section{Validation and Experimental Analysis} % (fold)
\label{sec:floats_figures_and_captions}

Validation of the implemented solution torwards a full-fledgeness solution

Performance: Throughput and Latency

Attestation latency (between cliend and Redis)

Resource allocation


Also: Redis(base) vs T-Redis
	  
	  Datasets (Dimension, extension) -> to explore with benchmark tools
	  
	  Typology of operations -> ratio read/write
	  							time of searches (by attributes)
	  							time of upload code


% \subsection{Inserting Figures Wrapped with text} % (fold)
% \label{ssec:inserting_images_wrapped_with_text}
% 
% You should only use this feature is \emph{really} necessary. This means, you have a very small image, that will look lonely just with text above and below.
% 
% In this case, you must use the \verb!wrapfiure! package.  To use \verb!wrapfig!, you must first add this to the preamble:
% 
% \begin{wrapfigure}{l}{2.5cm}
%   \centering
%     \includegraphics[width=2cm]{snowman-vectorial}
%   \caption{A snow-man}
% \end{wrapfigure}	
% 
% \noindent\verb!\usepackage{wrapfig}!\\
% This then gives you access to:\\
% \verb!\begin{wrapfigure}[lineheight]{alignment}{width}!\\
% Alignment can normally be either ``l'' for left, or ``r'' for right. Lowercase ``l'' or ``r'' forces the figure to start precisely where specified (and may cause it to run over page breaks), while capital ``L'' or ``R'' allows the figure to float. If you defined your document as twosided, the alignment can also be ``i'' for inside or ``o'' for outside, as well as ``I'' or ``O''. The width is obviously the width of the figure. The example above was introduced with:
% \lstset{language=TeX, morekeywords={\begin,\includegraphics,\caption}, caption=Wrapfig Example, label=lst:latex_example}
% \begin{lstlisting}
% 	\begin{wrapfigure}{l}{2.5cm}
% 	  \centering
% 	    \includegraphics[width=2cm]{snowman-vectorial}
% 	  \caption{A snow-man}
% 	\end{wrapfigure}	
% \end{lstlisting}

% subsection inserting_images_wrapped_with_text (end)

% section floats_figures_and_captions (end)

