%!TEX root = ../template.tex
%%%%%%%%%%%%%%%%%%%%%%%%%%%%%%%%%%%%%%%%%%%%%%%%%%%%%%%%%%%%%%%%%%%%
%% chapter2.tex
%% NOVA thesis document file
%%
%% Chapter with the template manual
%%%%%%%%%%%%%%%%%%%%%%%%%%%%%%%%%%%%%%%%%%%%%%%%%%%%%%%%%%%%%%%%%%%%
\chapter{Related Work}
\label{cha:users_manual}

% ================
% = Introduction =
% ================
Encryption of data stored in a computer is now more important than ever. Today we live in a world where almost every program we want to run or any application we want to use goes through a large number of different machines, therefore splitting our data through all. 

That is why it's needed to make sure all this information is unaccessible to potencial threats through encryption, regardless of which machine has the data. 

Although encrypting the stored data is a step forward towards the security of data, this is not enough. When one machine runs a program, the data needs to be decrypted and stored into RAM to be currectly executed, thus creating a potencial vulnerability. There is where Trusted Computing Environments come in.

We’ll look into more detail about existing technologies that approach this objective, and how they work or can work together as a way to grant confidentiality and integrity of data to the user.

(…)

In the summary, it’s expected to have a good notion of what has been developed in those areas to tackle this security concerns, as well as which of this thesis main points have already been addressed by previous work.

\section{Trusted  Computing Environments} % (fold)
\label{sec:introduction}
We're currently in a period where we start to depend more and more on allowing other services to do our work for us. Technologies like cloud computing require users to trust these systems. Therefore it's needed something to grant some respectable level of security, as well as being able to grant trust to the user. That's were Trusted Computing Environments (TCE) come in handy.
A TCE is a concept that came to grant integrity and confidentiality to systems by forcing a certain machine to behave an expected way, and deny any unwanted access to the data while decrypted. This way, even if the system does not run in a trusted machine, it can be expected that it will execute as it should. 

TCEs also protect the system against components that should always be trusted, like the host OS or even system admins, due to the priviledges they've got. This prevention avoids this components to abuse these priviledges, and thus always guarantee the normal execution of the system. 

In the following sub-chapters we'll see how to achieve this properties, what components can be used to do it and also how each of them works.

\subsection{TPM – Trusted Platform Modules }

A Trusted Platform Module (TPM), proposed by the Trusted Computing Group (TCG), is a set of hardware and software technologies that aim to create trust in a local platform \cite{sgxCloudThesis}. It is identified by an Endorsment Key-pair (EK) which is unique for every TPM, and also a Storage Root Key (SRK) that is used to protect other keys and data in the TPM. 

As for the hardware, a TPM consists mainly on a chip, found usually in the motherboard of most machines nowadays, that provides and stores cryptographic keys that can be used to grant integrity and data confidentiality to the system, as well as persistent and volatile memory to store these keys. 

As said before, the main objective of a TPM is to create the idea of a trusted platform. This will be  provided by three main services: Encryption, Authenticated Boot and Attestation. The first one is used for pretty much every aspect related with security and privacy. The Authenticated boot consists in booting the OS in stages, as a way of keeping track of which code is trustable through the usage of Platform Configuration Registers (PCR), that store the trusted software hashes. As for the Attestation, we'll see in the next subsection.


\subsection{TPM – Enabled Software Attestation}


TPMs enable the use of Remote attestation, which is the capability of one system to determine if other system can be trusted to run a particular piece of software as expected or not.

This is made possible by having a trusted configuration of state as reference, provided by the PCRs defined in the boot sequence, followed by a remote system that proceeds to challenge the trusted platform (containing the TPM) with a nonce. Then the platform creates a message with the nonce received previously and the existing configuration and calculates an hash value for that message. With an Attestation Identity Key (AIK), the message is then signed and sent back to the remote system, which then proceeds to decrypt the message with the EK public part, that then compares the result with the hash of the nonce plus the configuration it had at the begining of the challenge. 

If the hashes match, the remote system can then identify the TPM platform as a trusted platform.

\subsection{HSM – Hardware Security Modules}

Hardware Security Modules (HSM) are physical components whose main function is to provide and store cryptographic keys used to encrypt/decrypt data inside a system. HSMs can also perform cryptographic operations (e.g. encryption, hashing, etc.) as well as authenticate through verification of digital signatures and accelerate SSL connections \cite{hsmThesis}.

This modules are used mainly in large environment systems (e.g. large distributed systems) where there are a lot of machines communicating with each other, therefore creating a more needed sense of security. The inclusion of this modules in these big systems is actually a good idea since HSMs can also help servers relieve the workload coming from cryptographic operations. However HSMs don't quite guarantee the idea of absolute security, but increment the cost of attacking the system.

Although HSMs grant some extra level of security to a system, there's some drawbacks. One of them is the cost, where to buy one HSM the price varies depending on the sophistication of the security, plus the cost of maintenance makes it even more expensive. Another disadvantage is the difficulty on updating the module. Let's say a weakness was found in a cryptographic algorithm, it's hard to update the software in an already functional HSM to eliminate that weakness \cite{hsmPaper}.

\subsection{Trusted Execution Environments}

A Trusted Execution Environment (TEE) is an abstraction provided by both software and hardware that guarantees isolated execution of specific programs from other programs running on the same machine \cite{guiBorgesThesis}, but also from the host OS, hypervisor or even system administrators, preventing them from leveraging their priviledges and thus take advantage of the system. A TEE also grants secured storage of sensitive data, as well as remote attestation to make sure a given program runs as expected on a remote TEE. 

For a user to communicate with his program running inside an isolated environment, a key-exchange between the TEE and the user takes place. This way it is ensured both integrity and confidentiality of data during further communications.

This TEE abstraction can be achieved either by using a virtual machine monitor or by running security critical software (from whole applications to little segments of code) under protection mechanisms provided by hardware \cite{pcopThesis}. 

In the next chapters we will be looking into more dept about this hardware protection providers, that are responsable to create this trusted environments in the systems nowadays.

\subsection{Intel SGX}

Intel Security Guard Extensions (SGX) are a set of instructions built in Intel CPUs, that allow programmers to create TEEs, called enclaves. Enclaves are isolation containers that create an isolated environment where sensitive code can be stored and executed inside, ensuring integrity and confidentiality to it. By doing so, it reduces the Trust Computing Base, in a way that most of the system software, apart from the enclaves itself, is considered not trusted.

Enclaves are mapped into private regions of memory, where only the CPU as access to, reducing the TCB to only the enclave and the CPU itself. Due to this restrictions, not even the most common system libraries can be accessed inside the enclave, since the OS is not trusted. 

A system that encorporates the SGX under it's architecture is divided in two components, a trusted being the enclave, and an untrusted being the rest of the system. The untrusted component requests the launch of the enclave, where the CPU then manages to allocate the enclave in a private region of memory, made available only to that particular enclave. This portion of memory is kept encrypted in volatile memory, being only decrypted by the CPU if the responsible enclave requests for it \cite{guiBorgesThesis}.

Altough isolation is the main objective of Intel SGX, it still allows a way for both  untrusted and trusted parties to communicate. This is made possible by the functions ECALL and OCALL. ECALLs are used for an untrusted component to call for trusted code in a secured way - the enclave copies the pointers to that specific code into a buffer, who is then made visible for the untrusted component, ensuring that the untrusted party can't know the real memory address inside the enclave. To communicate the other way arround the enclave calls for an OCALL, where the enclave is temporarily exited, executing then the unstrusted function needed. After that, the enclave is re-entered. This is mainly used by the enclave to access the network or to deal with I/O disk access.


\subsection{RISC-V enabled Sanctum}

Just like Intel SGX, the main objective of Sanctum is to offer strong isolation of software modules, altough following a different approach focused in avoiding unnecessary complexity, thus granting a simple security analysis.
To make this possible, Sanctum, which tipically runs in a RISC-V processor, combines minimal and minimally invasive hardware modifications with a trusted software security monitor that is receptive to analysis and does not perform cryptographic operations using keys. 

This minimality idea consists on reusing and slightly modifying existing well-understood mechanisms, while not modifiying CPU building blocks, only adding hardware to the interfaces between blocks, causing Sanctum to be adaptable to other processors in addition to RISC-V \cite{sanctumPaper}. 

Sanctum is a pratical approach that shows that a strong software isolation is achievable with a small set of minimally invasive hardware changes, causing reasonably low overhead. 
This approach provides strong security guarantees dealing with side-channel attacks, such as cache timing and passive address translation attacks.

\subsection{ARM Trust Zone}

ARM TrustZone are hardware security extentions offered by ARM application processors with the same finality as Intel SGX, create insolated environments where software execute in a trustable way. 

To accomplish this, ARM processors implement two virtual processors backed by hardware access control, where the software stack can switch between two states called secure world (SW) and normal world (NW). 
The first one has with higher priviledges than the second one, therefore it can access NWs copies of registers, but not the other way around. SW is also responsible of protecting running processes in the CPU, while providing secured access to peripherals. 
Each world acts like a runtime environment and has it's own set of resources. This resources can be partioned between the two worlds or just assigned to one of them, depending on the ARM chip.

For the context switch between worlds, ARM processors implement a secured mode called Secure Monitor, where there's a special register responsable of determine if the processor runs code in SW or NW. 

Most ARM processors also offer memory curtaining. This consists on the Secure Monitor allocating physical addresses of memory specifically to the SW, making this region of memory unaccessible to the rest of the system.

By default, the system boots always in SW so it can provision the runtime environment before any untrusted code start to run. It eventually transitions to NW where untrusted code can start to be executed. \cite{armTZPaper}




\subsection{AMD SEV}

AMD Secure Encrypted Virtualization (SEV) is the AMD approach to provide a TEE integrated with virtualization. It's a technology focused primarily on cloud computing environments, specifically in public infrastructure as a Service (IaaS), as its main goal is to reduce trust from higher priviledged parties (VMMs or OS), so that they can not influence the execution on the other "smaller" parties (VMs). 

To achieve this, AMD grants encryption of memory through a technology called Secure Memory Encryption (SEM), or through TransparentSEM (TSEM) if the system runs a legacy OS or hypervisor with no need for any software modifications.
After the data is encrypted, SEV integrates it with AMD virtualization architecture to support encrypted VMs. By doing this, every VM is now protected from his own hypervisor (VMM), unabling its access to the decrypted data. Although uncapable of accessing the VM, the VMM is still responsible of controlling each VM resources. \cite{amdPaper}

Thus, AMD provides confidentiality of data by removing trust from the VMM, and creates an isolated environment for the VM to run, where only the VM and the processor can be trusted. However it does not provide integrity of data, allowing replaying attacks to take place, and has a considerably large TCB, since the OS of each VM is trusted. \cite{amdSEVPaper}


\subsection{Discussion}

Write some comparison conclusions of why SGX and not the others.



% section introduction (end)

% ====================
% = Folder Structure =
% ====================
\section{TEE/SGX Enabled Protection Against Untrusted OSes} % (fold)
\label{sec:folder_structure}

\subsection{Virtual Ghost}



\subsection{Flicker}
\subsection{MUSHI}
\subsection{SeCage}
\subsection{InkTag}
\subsection{Sego}
\subsection{Other approaches}
\subsection{Discussion}


The \novathesis\ template is organized into files and folders. At the main level it includes the following files and folders:

\noindent
\begin{tabularx}{\linewidth}{>{\ttfamily}l>{\itshape}l>{\upshape}X}
novathesis.cls     & file    & 
The main class file. It will include additional files from \texttt{novathesis-files} folder. 
\\ 
template.tex      & file    & 
The main user file. Use this file as the main file for your thesis. 
\\
bibliography.bib  & file    & 
An example of a bibliography file. You may have has many as you want. \\
template.pdf      & file    & 
A possible result of applying pdf\LaTeX\ to the \texttt{template.tex} file. The template supports multiple types of documents (e.g., MSc dissertation, PhD thesis, …) and multiple Schools (e.g., FCT-NOVA, FCSH-NOVA, IST-UL, FC-UL, …) and each will produce different results.
\\
Chapters          & folder  & Examples of thesis chapters. Replace them with your own chapters. 
\\
Examples          & folder  & Some more examples of the use of the template for different document types and Schools. 
\\
Scripts           & folder  & Some (possibly useful) scripts for Unix-based systems (Linux, Mac OSx). If you are a windows user, ignore this folder (you may safely delete it if you want). 
\\
novathesis-files   & folder  & 
Additional files for the \novathesisclass\ file.  Unless you know what you are doing, avoid messing up with the files and folders inside this folder (except for deleting the unused Schools, see below). 
\\
\end{tabularx}

The \texttt{novathesis-files} folder contains additional files and folders that complement the main \novathesisclass\ file.  These are:

\noindent
\begin{tabularx}{\linewidth}{>{\ttfamily}l>{\itshape}l>{\upshape}X}
README.txt      & file    &
A file that should be read!  :) 
\\
fix-babel.clo   & file    &
Simple fixes to the \texttt{babel} package.
\\
lang-text.clo   & file    &
Translations of important strings used in the template.  Currently fully supported are Portuguese and English, but French is on the way.  If you add translations for your own language, please be so kind and send them to me. Thank you!
\\
options.clo     & file    &
Processing of \novathesisclass\ options.  \emph{Don't mess with this!}
\\
packages.clo    & file    &
Additional packages to be loaded into the \novathesis\ template. \emph{You should not mess with this!}
\\
spine.clo       & file    &
This file is loaded only if the option \texttt{spine=true}, and includes the typesetting of the book spine.
\\
ChapStyles      & folder  &
Contains a lot of files, one for each chapter style.  If you really know what you are doing, you may add your own chapter style here.
\\
FontStyles      & folder  &
Contains a few files, one for each set of fonts (main text font, chapter font, section font, subsection font, etc).  If you really know what you are doing, you may add your own set here.
\\
Schools         & folder  &
Configuration files for each school.  This folder is organized into subfolders, one for each university.  \emph{You may safely delete all the subfolders except the one for your University.}  Then open the subfolder of your University and \emph{you may safely delete all the subfolders except the one for your School/Faculty}.
\\
\end{tabularx}

As stated above, the \texttt{Schools} folder contains per-university folders and per-school (faculty) subfolders.  Currently these are the available folders:

\noindent
\begin{tabularx}{\linewidth}{>{\ttfamily}r@{~/~}>{\ttfamily}l>{\itshape}l>{\upshape}X}
ul     & ist    & folder  & 
The folder for the \href{http://www.tecnico.ulisboa.pt}{\emph{Instituto Superior Técnico}} of the \emph{University of Lisbon}.
\\
nova    & fcsh   & folder  & 
The folder for the \href{http:www.fcsh.unl.pt}{\emph{Faculty of Human and Social Sciences}}  of the \emph{NOVA University of Lisbon}.
\\
nova    & fct    & folder  & 
The folder for the \href{http:www.fct.unl.pt}{\emph{Faculty of Sciences and Technology}} of the \emph{NOVA University of Lisbon}.
\\
nova    & novaims    & folder  & 
The folder for the \href{http:www.novaims.unl.pt}{\emph{Information and Management School}} of the \emph{NOVA University of Lisbon}.
\\
\end{tabularx}

% section folder_structure (end)

% ===================
% = Package options =
% ===================
\section{SGX-Frameworks and Application Support} % (fold)
\label{sec:package_options}
\subsection{VC3 protection for MapReduce Jobs}
\subsection{Protected Zookeeper}
\subsection{Ryoan Sandboxing}
\subsection{Opaque}
\subsection{Graphene-SGX}
\subsection{Network services protection approaches}
\subsection{Application-level protection approaches}
\subsection{Discussion}


The \novathesis\ class can be customized with the options listed below.

\newcommand{\classoption}[3]{\textbf{#1=OPT}\qquad #2\\\qquad\emph{#3}\\}

\noindent
\begin{ctabular}{@{}p{\linewidth}@{}}
  \toprule
  \classoption{docdegree}%
    {phd(*), phdplan, phdprop, msc, mscplan, bsc}%
    {The type of the document: PhD Thesis (default), PhD Plan, PhD Proposal, MSc Disseration, MSc Plan, BSc Report}
    \midrule
  \classoption{school}%
		{nova/fct(*), nova/fcsh, nova/ims, ul/ist, ul/fc}%
    {The name of the school. This option changes the typesetting of the cover and some School specific formating, like margins, fonts, paragraph spacing and indentation, etc…}
    \midrule
  \classoption{lang}%
    {en(*), pt}%
    {The main language for the document.  Currently only Portuguese and English are supported.  Other languages are expected to be support in forthcoming versions.}
    \midrule
  \classoption{fontstyle}%
    {bookman, charter, fourier, kpfonts(*), mathpazo1, mathpazo2, newcent}%
    {The font set to be used in the document.  Please note that a font set include definitions for the main text, headings, maths, etc.}
    \midrule
  \classoption{chapstyle}%
    {bianchi, bluebox, brotherton, dash, default, elegant(*), ell, ger, hansen, ist, jenor, lyhne, madsen, pedersen, veelo, vz14, vz34, vz43}%
    {The chapter style, i.e., the look of the chapter beginning.}
    \midrule
  \classoption{converlang}%
    {en, pt(*)}%
    {The language to be used when typesetting the cover page.}
    \midrule
  \classoption{otherlistsat}%
    {front(*), back}%
    {Where to put the other lists besides the table of contents. The default is (\texttt{front}) before the main text.  But some scientific areas prefer them at the end of the document (\texttt{back}), just before the Appendixes.}
    \midrule
  \classoption{aftercover}%
    {true, false(*)}%
    {Include or don't include the contents of the “\texttt{aftercover}” file. The default is for this file to be ignored (if if it exists).}
    \midrule
  \classoption{linkscolor}%
    {darkblue(*), black}%
    {The color for all the hyperlinks in the PDF file.  The “\texttt{media=paper}” option (see below) will override this option to “\texttt{black}”}
    \midrule
  \classoption{spine}%
    {true, false(*)}%
    {Generate the book spine and the last page in the PDF.}
    \midrule
  \classoption{biblatex}%
    {OPT=\{list of options for \texttt{biblatex}\}}%
    {Customize \texttt{biblatex}, the bibliography management system used in this class. Probably you will want to change the value of the \texttt{biblatex} “\texttt{style}” option. For other customizations of \texttt{biblatex} check its manual.}
    \midrule
  \classoption{memoir}%
    {OPT=\{list of options for \texttt{memoir}\}}%
    {Customize the base class \texttt{memoir}. The \texttt{memoir} manual should be the first document to be consulted when looking for “\textbf{how can I do this?}” You may wnat to change the base font size from 11pt to a smaller (10pt) or larger (12pt) size.  Also, remember to change the “\texttt{draft}” to final when your document is finished.}
    \midrule
  \classoption{media}%
    {screen(*), paper}%
    {Behavior to be customized in the school options/configuration. Expected definitions for screen are: left and right margins are equal and use colored links. Expected definitions for paper are: left and right margins are different and use black links.}
    \bottomrule
\end{ctabular}

\section{Related work analysis and rational} % (fold)
\label{sec:additional_considerations}

In this section we will provide some additional considerations about some of the customizations available as class options.



