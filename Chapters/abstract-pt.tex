%!TEX root = ../template.tex
%%%%%%%%%%%%%%%%%%%%%%%%%%%%%%%%%%%%%%%%%%%%%%%%%%%%%%%%%%%%%%%%%%%%
%% abstrac-pt.tex
%% NOVA thesis document file
%%
%% Abstract in Portuguese
%%%%%%%%%%%%%%%%%%%%%%%%%%%%%%%%%%%%%%%%%%%%%%%%%%%%%%%%%%%%%%%%%%%%

	O Intel SGX é uma base de computação confiável para que aplicações se protejam de SOs ou Hipervisores potencialmente maliciosos. Em \textit{cloud} ou noutros ambientes de computação garantidos por terceiros, muitos utilizadores e aplicações podem beneficiar do SGX. Trabalhos já realizados que abordaram a emulação de bibliotecas de SOs visaram a execução de aplicações não modificadas sobre SGX, mas surgiu uma convicção de que esse tipo de abordagem não será interessante no que toca à performance ou ao tamanho da base de computação confiável fazendo com que, na prática, tanto modificações como a reengenharia do código de aplicações estejam sempre implícitos e sejam um melhor pré-requisito para adotar ambientes de computação com SGX.
	
	(HELP) this next paragraph enters the abstract? or just conclusions after it's done?
	
	Nesta tese pretendemos estudar as abordagens já existentes de bibliotecas de SOs e conduzir uma avaliação experimental adotando a solução recente de uma emulação de uma biblioteca de SOs para ser usada sobre o SGX, como sendo uma biblioteca completamente caracterizada que possa eventualmente ser adotada para implantar aplicações não modificadas de forma rápida, com \textit{overheads} comparáveis a aplicações modificadas para usar camadas \textit{"shim"}. O foco da nossa avaliação consistirá na virtualização da base de dados do tipo Chave-Valor  \textit{Redis}, redesenhando e implementando-a como uma base de dados Chave-Valor confiável com permissões SGX (TREDIS).
	
% Palavras-chave do resumo em Português
\begin{keywords}
Intel SGX, REDIS, Ambiente de Execução Confiável, Proteção de Dados, Privacidade, Confiabilidade \ldots
\end{keywords}


	
