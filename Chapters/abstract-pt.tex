%!TEX root = ../template.tex
%%%%%%%%%%%%%%%%%%%%%%%%%%%%%%%%%%%%%%%%%%%%%%%%%%%%%%%%%%%%%%%%%%%%
%% abstrac-pt.tex
%% NOVA thesis document file
%%
%% Abstract in Portuguese
%%%%%%%%%%%%%%%%%%%%%%%%%%%%%%%%%%%%%%%%%%%%%%%%%%%%%%%%%%%%%%%%%%%%
A transição de suporte de aplicações com armazenamento e processamento em servidores \textit{cloud} é uma tendência que tem vindo a aumentar, principalmente quando se precisam de gerir grandes conjuntos de dados. Comparativamente a soluções com licenciamento privado, as soluções de computação e armazenamento de dados em nuvens de serviços são capazes de oferecer opções mais baratas, de alta disponibilidade, elásticas e relativamente confiáveis. Estas soluções fornecidas por terceiros são facilmente acessíveis através da Internet, sendo operadas em regime de \textit{outsourcing} da sua operação, o que é bom, mas que por isso ficam consideravelmente expostos a ataques e fora do controle dos utilizadores em relação às reais condições de confiabilidade, segurança e privacidade de dados. Ao explorar subtilmente vulnerabilidades presentes nas aplicações, funções de sistemas operativos (SOs), bibliotecas de virtualização de serviços de SOs ou hipervisores, um atacante pode comprometer os sistemas e quebrar a privacidade de dados sensíveis. Estes ataques podem ser motivados por fins maliciosos como espionagem, chantagem, roubo de identidade ou assédio e podem ser desencadeados por intrusões (a partir de atacantes externos) ou por ações maliciosas ou incorretas de atacantes internos (podendo estes atuar com privilégios de administradores de sistemas). Uma solução para este problema passa por armazenar e processar a informação sem que existam exposições face a componentes não confiáveis. 

Nesta dissertação estudamos e avaliamos experimentalmente diversas tecnologias que permitem a execução de aplicações com isolamento em ambientes de execução confiável suportados em hardware Intel-SGX, de modo a perceber melhor como funcionam e como adaptá-las à nossa solução. Para isso, realizámos uma avaliação focada na utilização dessas tecnologias com virtualização em contentores isolados executando em hardware confiável, que usámos na concepção da nossa solução. Posto isto, apresentamos a nossa solução TREDIS - um sistema \textit{Key-Value Store} confiável baseado em tecnologia REDIS, com garantias de integridade da execução e de privacidade de dados, concebida para ser usada como uma "Plataforma como Serviço" para gestão e armazenamento resiliente de dados na nuvem. Isto inclui a possibilidade de suportar uma arquitetura segura com garantias de resiliência semelhantes à arquitetura de replicação em \textit{cluster} na solução original REDIS, mas em que os motores de execução de nós e a proteção de memória do \textit{cluster} é baseado em contentores docker isolados e virtualizados em instâncias SGX, sendo os dados mantidos sempre cifrados em memória.
	
% Palavras-chave do resumo em Português
\begin{keywords}
 Intel SGX, REDIS, Computação Confiável, Ambientes de Execução Confiáveis, Proteção de Dados, Preservação de Privacidade, \textit{Key-Value Stores} Confiáveis em Memória
\end{keywords}


	
