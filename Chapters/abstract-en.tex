%!TEX root = ../template.tex
%%%%%%%%%%%%%%%%%%%%%%%%%%%%%%%%%%%%%%%%%%%%%%%%%%%%%%%%%%%%%%%%%%%%
%% abstrac-en.tex
%% NOVA thesis document file
%%
%% Abstract in English
%%%%%%%%%%%%%%%%%%%%%%%%%%%%%%%%%%%%%%%%%%%%%%%%%%%%%%%%%%%%%%%%%%%%

Intel SGX hardware is a trust-computing base for applications to protect themselves from potentially-malicious OSes or hypervisors. In cloud and other outsourced computing environments, many users and applications could benefit from SGX. 
However, legacy applications are not prepared to work out-of-the-box on SGX. 

Previous research work have already addressed library emulated OSes targeted to execute unmodified applications on SGX, but a belief has emerged that such approaches will not be interesting in terms of performance and TCB size, making in practice that application code modifications or reengineering is always an implicit and better prerequisite for adopting SGX-enabled computing environments. 

(HELP) this next paragraph enters the abstract? or just conclusions after it's done?

In this thesis we intend to study existent library OSes approaches and conduct an experimental evaluation by adopting a recent solution of a OS library emulation to be ported on top of SGX, as a fully-featured library OS that can eventually be adopted to rapidly deploy unmodified applications, with overheads comparable to applications modified to use “shim” layers. Our targeted evaluation will be conducted in virtualizing the Redis Key-Value Store, redesigning and implementing it as a SGX-enabled Trusted Key-Value Store (TREDIS).
% Palavras-chave do resumo em Inglês
\begin{keywords}
Intel SGX, REDIS, Trusted Execution Environment, Data Protection, Privacy, Dependability \ldots
\end{keywords} 


